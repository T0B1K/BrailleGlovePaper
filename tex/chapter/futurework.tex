\chapter{Conclusion and Future Work}
\label{ch:conclusion}

%SUMMARY
We conclude the first study with that the tapping and vibration stimuli performed similarly overall, while the stroking stimulus appeared to perform slightly worse. Based on both the quantitative and qualitative data, we found no statistically significant difference between two-handed \gls{ost} encoding for affective versus discriminative touch stimuli. Nevertheless, there is a small difference indicating that both the tapping and vibration stimuli were slightly more effective, both in terms of performance and the perceived usefulness reported by participants.
%Second Study----
For the secon study, there are no significant differences between the encodings. However, the \gls{seq} encoding was slightly better than the \gls{ost} encoding in the tests, and the comments were more positive regarding \gls{seq}. Nevertheless, the direct comparison revealed that there is no large difference between the two encodings. Some participants expressed strong opinions against \gls{seq} due to its distinguishability, while others preferred it for being more comfortable. This suggests that both encodings are valuable for passive haptic learning. Additionally, there appears to be no substantial difference in the number of repetitions between the two encodings. Although the \gls{ost} encoding involved more repetitions, the \gls{seq} encoding did not result in a higher number of repetitions within the same time frame, as the \gls{ost} encoding was faster. Therefore, we conclude that time, rather than the number of repetitions, is the key difference between the two encodings.
%SUMMARY-----
In summary we showed that some questionnaire dimensions were correlated across both studies. While this does not imply causality—more evidence is needed—it remains an interesting finding.

We also demonstrated that both discriminative and affective touch methods are compatible with two-handed chorded input. However, since there was no significant improvement, we reject our RQ1.

As the first paper in this area, we further showed that different encoding schemes between the two hands are possible and do not significantly differ in terms of braille learning. This led us to reject RQ2. However, it opens up more experimental options for how to encode chords.

We also showed that chorded input using non-vibration stimuli is a viable approach for learning braille through \gls{phk}. Finally, we provided a system design that supports two-handed chorded input in PHL, which can be used for future experiments on braille learning and other two-handed skills.

In future studies, tapping as a stimulus could be further explored for passive haptic learning, as it is not yet commonly utilized. Additionally, the interaction between audio and vibration offsets, as well as the length of stimulus activation, presents an interesting area for investigation. The current setup could also be adapted to explore the potential for learning two-handed instruments, such as the flute, using passive haptic learning with chorded pieces, or other activities that require multi-finger coordination.

Another direction for future research involves testing the \gls{seq} encoding with the tapping stimulus to determine whether the combination of these two factors yields better learning outcomes compared to other encoding methods. While our current findings suggest that time, rather than the number of repetitions, is the key factor in the efficacy of different encodings, further studies are needed to explore this relationship in more detail.

Additionally, the relatively low error rate in single-finger tasks opens the possibility of using artificial intelligence to predict braille characters, potentially reducing the overall learning time. It would also be valuable to explore whether different teaching methods can better reinforce muscle memory in braille learning using \gls{phl}. These avenues may offer deeper insights into optimizing passive haptic learning systems for a range of learning contexts.
% %--- take away---
% Moreover, we showed that some questionnaire dimensions were correlating for both of the studies. Even though this doesn't mean causality, as more evidence is needed, it is still an interesting finding.
% Moreover, we showed, that both discriminative and affective touch methods work for two handed chorded input, however, there is no significant improvement, herefore we reject our RQ1.
% Additionally as the first paper in this area, we found out, that different encoding schemese between two hands are indeed possible and don't differ signifficantly in learning braille, which led us to reject RQ2 but offers experiments more options in dealing with the problem of encoding chords.
% Moreover, we showed, that chorded input on non-vibration stimuli is a viable option for learning braille using phl and lastly we provide a system design, that can be used for two-handed, chorded input systems in phl for further tests on passive haptic braille learning and other two handed skills.
% %-----take away
% In future studies, tapping as a stimulus could be further explored for passive haptic learning, as it is not yet commonly utilized. Additionally, the interaction between audio and vibration offsets, as well as the length of stimulus activation, presents an interesting area for investigation. The current setup could also be adapted to explore the potential for learning two-handed instruments, such as the flute, using passive haptic learning with chorded pieces or even other activities that require multi-finger coordination.

% Another direction for future research involves testing the \gls{seq} encoding with the tapping stimulus to determine whether the combination of these two factors yields a slightly better learning outcome compared to other encoding methods. While our current findings suggest that time, rather than the number of repetitions, is a key factor in the efficacy of different encodings, further studies are needed to investigate this relationship in greater detail.

% Additionally, the relatively low error rate in single-finger tasks opens up the possibility of employing artificial intelligence to predict the braille characters, which could potentially shorten the learning time for braille. It would also be valuable to explore whether different teaching methods can better drill muscle memory for braille learning using \gls{phl}. These avenues could provide deeper insights into the optimization of passive haptic learning systems and their application in diverse learning contexts.

