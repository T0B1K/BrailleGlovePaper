\chapter{Introduction}
\label{ch:Introduction}

Worldwide, approximately 285 million people are visually impaired (with less than 1.3% vision), and about 39 million are completely blind, according to WHO \cite{Pascolini2012}.
In Germany alone, there are about 1.066k visually impaired individuals and approximately 160 thousand that are blind, according to Bertram et al. \cite{Bertram2005}.
Learning Braille remains a significant challenge for blind and visually impaired individuals. While Grade 1 Braille can be learned in a few months, Grade 2 Braille takes more than a year to master. According to the Commission for the Blind of New York State \url{https://ocfs.ny.gov/programs/nyscb/assets/docs/BrailleFAQ.pdf}. Learning Braille is particularly difficult for those who lose their vision later in life \cite{Seim2014a}.
Braille is more than just a reading system; it is a critical literacy tool directly linked to better educational outcomes and employment opportunities \cite{Seim2014a, Ryles1996}. Ryles et al. note that Braille is the primary medium of literacy for blind individuals and is associated with solid reading habits and a greater likelihood of pursuing post-secondary education \cite{Ryles1996}. Furthermore, Bell et al. \cite{Bell2013} showed, that the daily use of Braille positively impacts employment, salary, and self-esteem.
Despite these clear benefits, only 10% of blind individuals learn Braille due to a shortage of certified instructors and resources. As a result, 74% of blind individuals are unemployed \cite{Seim2014a}.
The challenges of Braille literacy are further compounded by the wide variability of teaching approaches, each requiring learners to have ample opportunities to use and develop their understanding of Braille contractions \cite{Swenson1999}. Modern technologies such as text-to-speech are able to help visually impaired people participate in active life but also cause neglect in Braille instruction \cite{Seim2014a}. However, excessive reliance on audio learning can negatively affect essential literacy skills such as spelling and composition \cite{Foulke1979}, as full reliance on audio technology is inconsistent with the broader definition of literacy, which also includes writing \cite{tuttle1996point}.
In light of these barriers to traditional Braille instruction, several Braille learning devices such as gloves have been invented like the devices by Zaman et al. \cite{Zaman2019}, Ozioko et al. \cite{Ozioko2017}, An \cite{An2004} Cho et al. \cite{Cho2002} and other such as Seim et al. \cite{Seim2014a, Seim2014, Seim2015}, Yang et al. \cite{Yang2017} or Forsyth et al.\cite{Learning2024}.
And also different teaching / acquiring methods such as active and passive learning. In our opinion, \gls{phl} offers a promising solution due to the lack of active attention needed. \Gls{phl} enables individuals to learn motor tasks, such as Braille typing, through tactile stimulation without requiring active attention. Research teams have leveraged this concept to design assistive gloves that teach Braille through muscle memory, a form of procedural memory that allows individuals to unconsciously retain skills over time \cite{Yang2017, Seim2015}. By using \gls{phl}, individuals can passively learn Braille while performing everyday activities, such as walking or commuting, reducing the time and effort required for instruction \cite{Yang2017}.
The potential of \gls{phl} to accelerate Braille learning is further supported by research into the neuroplastic changes that occur during the process. Blind individuals who learn Braille experience cortical changes, including enlargement of the sensorimotor area associated with the reading finger and the recruitment of the occipital cortex, which formerly processed visual information, for tactile tasks \cite{Hamilton1998a}. By providing continuous, passive tactile stimulation, \gls{phl} systems may enhance these neuroplastic processes, making Braille learning more efficient. In doing so, \gls{phl} offers a viable solution to address the "crisis" in Braille literacy \cite{Seim2014a}, providing an innovative method for improving literacy and independence for those who are blind or visually impaired.
However, many open questions remain in this domain, such as: \textit{\textbf{RQ1:} Is there a difference between affective and discriminative touch for both hands using the \gls{ost} encoding?} and \textit{\textbf{RQ2:} Is there a significant difference between using the \gls{ost} and the \gls{seq} encoding?} These are the questions this thesis aims to address, with the goal of easing the burden of learning Braille and enabling more individuals to embrace a life of literacy."


