\chapter{Conclusion and Future Work}
\label{ch:conclusion}


In future studies, tapping as a stimulus could be further explored for passive haptic learning, as it is not yet commonly utilized. Additionally, the interaction between audio and vibration offsets, as well as the length of stimulus activation, presents an interesting area for investigation. The current setup could also be adapted to explore the potential for learning two-handed instruments, such as the flute, using passive haptic learning with chorded pieces or even other activities that require multi-finger coordination.

Another direction for future research involves testing the \gls{seq} encoding with the tapping stimulus to determine whether the combination of these two factors yields a slightly better learning outcome compared to other encoding methods. While our current findings suggest that time, rather than the number of repetitions, is a key factor in the efficacy of different encodings, further studies are needed to investigate this relationship in greater detail.

Additionally, the relatively low error rate in single-finger tasks opens up the possibility of employing artificial intelligence to predict the braille characters, which could potentially shorten the learning time for braille. It would also be valuable to explore whether different teaching methods can better drill muscle memory for braille learning using \gls{phl}. These avenues could provide deeper insights into the optimization of passive haptic learning systems and their application in diverse learning contexts.

