\chapter{Conclusion and Future Work}
\label{ch:conclusion}

%SUMMARY
We conclude the first study with that the tapping and vibration stimuli performed similarly overall, while the stroking stimulus appeared to perform slightly worse. Based on both the quantitative and qualitative data, we found no statistically significant difference between two-handed \gls{ost} encoding for affective versus discriminative touch stimuli. Nevertheless, there is a small difference indicating that both the tapping and vibration stimuli were slightly more effective, both in terms of performance and the perceived usefulness reported by participants.
%Second Study----
For the secon study, there are no significant differences between the encodings. However, the \gls{seq} encoding was slightly better than the \gls{ost} encoding in the tests, and the comments were more positive regarding \gls{seq}. Nevertheless, the direct comparison revealed that there is no large difference between the two encodings. Some participants expressed strong opinions against \gls{seq} due to its distinguishability, while others preferred it for being more comfortable. This suggests that both encodings are valuable for passive haptic learning. Additionally, there appears to be no substantial difference in the number of repetitions between the two encodings. Although the \gls{ost} encoding involved more repetitions, the \gls{seq} encoding did not result in a higher number of repetitions within the same time frame, as the \gls{ost} encoding was faster. Therefore, we conclude that time, rather than the number of repetitions, is the key difference between the two encodings.
%SUMMARY-----

In future studies, tapping as a stimulus could be further explored for passive haptic learning, as it is not yet commonly utilized. Additionally, the interaction between audio and vibration offsets, as well as the length of stimulus activation, presents an interesting area for investigation. The current setup could also be adapted to explore the potential for learning two-handed instruments, such as the flute, using passive haptic learning with chorded pieces or even other activities that require multi-finger coordination.

Another direction for future research involves testing the \gls{seq} encoding with the tapping stimulus to determine whether the combination of these two factors yields a slightly better learning outcome compared to other encoding methods. While our current findings suggest that time, rather than the number of repetitions, is a key factor in the efficacy of different encodings, further studies are needed to investigate this relationship in greater detail.

Additionally, the relatively low error rate in single-finger tasks opens up the possibility of employing artificial intelligence to predict the braille characters, which could potentially shorten the learning time for braille. It would also be valuable to explore whether different teaching methods can better drill muscle memory for braille learning using \gls{phl}. These avenues could provide deeper insights into the optimization of passive haptic learning systems and their application in diverse learning contexts.

