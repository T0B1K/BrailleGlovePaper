\section{Related Works}
\subsection{Piano Learning}
In 2008, Huang et al. \cite{Huang2008} investigated the effectiveness of learning piano through passive haptic feedback combined with audio cues. A study was conducted utilizing a specialized glove embedded with vibration motors corresponding to each finger, designed to assess whether passive exposure to a combination of auditory and tactile stimuli could improve piano learning and performance relative to learning through auditory stimuli alone. Participants wore the glove while engaging in a 30-minute distraction task, such as reading or typing, during which piano music played and corresponding finger vibrations were delivered. Afterward, they were asked to play the piano pieces they had been exposed to. The results revealed that participants who received both audio and haptic feedback performed significantly better, playing the pieces more smoothly and with fewer errors compared to those who received only audio feedback. The latter group exhibited more hesitation and confusion. The study concluded that \gls{phl} can effectively enhance piano skill acquisition by reducing errors and improving performance fluidity. The combination of audio and tactile feedback provided a richer understanding of the musical structure, reinforcing the potential of \gls{phl} as a valuable tool for learning physical skills, such as piano playing.

Further investigation into this phenomenon was conducted within the \gls{mmt} framework. The first work by \cite{Huang2010} explored the potential of using \gls{phl} to teach piano playing, focusing on whether individuals could learn piano passages through haptic feedback delivered via fingerless vibration gloves and how this method compared to traditional active learning methods.
Two studies were conducted. In the first study, novice participants were asked to learn and reproduce a musical passage after a 30-minute session of \gls{phl} using the \gls{mmt}  system. During the session, participants engaged in a reading comprehension task while receiving tactile cues via the gloves. The results showed that participants who received tactile stimulation performed better than those in the control group, demonstrating that \gls{phl} can effectively teach piano passages. Interestingly, participants without prior musical experience tended to perform better with \gls{phl} than those with musical backgrounds. In the second study, the researchers compared the time required to learn a short, randomly generated musical passage using either passive or active learning methods. Participants with no prior piano experience often succeeded in repeating the passage correctly after \gls{phl}, while those with musical experience found the process more challenging. This study highlighted the advantages of \gls{phl} for novices, as it required fewer attempts to learn the passage compared to active learning, particularly for those without a musical background.
The paper also found that while audio alone was insufficient for effective learning, the combination of tactile feedback and passive exposure to musical sequences significantly improved participants' ability to learn and reproduce the music.

The second \gls{mmt} paper reported by \cite{Kohlsdorf2010} examined the impact of different primary tasks on the effectiveness of \gls{phl} in teaching piano note sequences. Three primary tasks were examined while participants received passive haptic feedback to learn a random 10-note piano sequence: watching a film (audio-visual stimulation), playing a memory game (active memory engagement), and walking a designated path (body motion). These tasks represented common everyday activities like relaxing, thinking, and walking. Participants initially observed and listened to the piano keyboard as it played a 10-note sequence, after which they attempted to replicate the sequence. If they were unable to do so successfully, they proceeded to engage in the primary task for five minutes while receiving targeted finger stimulation via the \gls{phl} device. Following this stimulation, participants made another attempt to reproduce the sequence. This cycle was repeated until the participant was able to accurately replicate the sequence with 100\% accuracy. The results indicated no statistically significant difference in the number of \gls{phl} sessions required across the different primary tasks. However, individual differences were noted, with some participants finding certain tasks more challenging than others. Rhythm retention was slightly better after the memory game condition compared to the film condition, though no condition was significantly better or worse overall. Participants reported similar subjective workloads across all conditions, suggesting that the type of primary task does not significantly impact the effectiveness of \gls{phl} in learning piano sequences. The study concluded that while the type of primary task may influence individual performance, there is no clear evidence that one task is superior to others for \gls{phl} effectiveness. This suggested that \gls{phl} can be effectively integrated into various everyday activities without being significantly hindered by the nature of the concurrent task.

Based on the aforementioned results, \cite{Seim2015b} investigated the potential of using \gls{phl} to teach two-handed, chorded piano melodies, focusing on the effectiveness of \gls{phl} in teaching complex motor skills and the necessity of accompanying audio feedback. The study introduced a method of sequentially delivering tactile stimulation to each finger involved in a chord, rather than simultaneous stimulation which has been proven to be ineffective, to facilitate the learning of complex piano pieces\cite{Luzhnica2017,Luzhnica2016,Luzhnica2018,Luzhnica2018a,Seim2014a}.
The study comprised two experiments. In the first, participants were exposed to vibration-only and vibration-plus-audio conditions while learning a simple one-handed piano melody. Their performance, assessed using Dynamic Time Warping (DTW), indicated no significant difference between the two conditions, suggesting audio feedback was not essential for effective learning with \gls{phl}.
In the second experiment, participants without prior piano experience were randomly assigned to one of the following four conditions: no feedback; audio only; vibration only; and audio with vibration. Those in the vibration-only and vibration-plus-audio groups performed significantly better than the control and audio-only groups, indicating the effectiveness of tactile feedback in learning motor skills. However, participants reported higher frustration levels when audio was combined with vibration, as measured by the NASA Task Load Index (NASA TLX).
The study concluded that \gls{phl} is effective for teaching two-handed, chorded piano melodies, particularly with tactile feedback alone. These findings suggest that passive haptic stimulation can suffice for learning motor skills, even with minimal attention dedicated to the task. The research also has broader implications for teaching other complex motor skills, like Braille typing, through \gls{phl}, and reinforced that \gls{phl} is most effective when teaching sets of 10-17 stimuli at a time, as explored in previous work \cite{Seim2014a}.

In $2021$, Donchev et al. \cite{Donchev2021} explored the effectiveness of \gls{phl} in teaching and retaining piano note sequences, comparing it to active learning methods. The study aimed to determine whether \gls{phl} could be as effective as active learning for memorizing and recalling piano sequences and to assess the long-term retention of these sequences.
Participants were taught a 10-note piano sequence through both active and \gls{phl} methods and were then tested three days later to evaluate their retention of taught sequences. The study design was based on previous \gls{phl} research, particularly the \gls{mmt} tactile stimulation method, but with modifications to prevent overlearning, as participants were stopped once they achieved 90\% accuracy during the learning phase.
The results indicated no significant difference in unaided recall between actively and passively learned note sequences when participants were asked to play from memory. However, when provided with auditory and visual cues, participants were significantly better at recalling the passively learned sequences. This suggested that while both learning methods are effective, \gls{phl} may offer advantages in cued recall situations. The study also observed a recency effect in \gls{phl}, where more recently learned material was better retained.
Moreover, the study found that while \gls{phl} took longer initially, participants required fewer attempts to learn the sequences and demonstrated a 10\% higher retention rate compared to active learning. This suggested that \gls{phl} can be a highly effective method for teaching and retaining complex skills like piano playing, especially when cues are provided during recall.

In $2023$, \ea{Fang}{\cite{Fang2023a}} investigated the impact of \gls{phl} on rhythm learning across different musical instruments, focusing on the keyboard and ukulele. The study also examined whether combining haptic feedback with audio enhances rhythm learning, as measured by accuracy in duration and timing.
Participants were divided into three groups: one receiving haptic feedback only, another receiving both haptic and audio feedback, and a third receiving audio feedback only. Inspired by previous research on \gls{phl} in piano learning\cite{Donchev2021, Huang2010}, the study offset the haptic and auditory signals by a few seconds to avoid overlap. The results indicated that the effectiveness of learning varied depending on the feedback method and the instrument used. For rhythm duration accuracy, the group that received only haptic feedback performed the best on the keyboard, followed by the group that received both haptic and audio feedback. However, performance across all groups was notably poor on the ukulele, suggesting that haptic feedback alone may be insufficient for learning more complex instruments.
Regarding timing accuracy, rhythms played on the keyboard were generally reproduced more accurately, with the group receiving both haptic and audio feedback achieving near-perfect timing. In contrast, the group relying solely on haptic feedback demonstrated the greatest difficulty with timing accuracy, particularly on the ukulele.Interviews with participants further revealed that the combination of haptic and audio feedback was advantageous for rhythm learning, especially when using the keyboard. However, relying exclusively on haptic feedback was found to be the least effective approach, particularly when applied to more complex instruments like the ukulele. The study concluded that while \gls{phl} can be effective for rhythm learning, its efficacy is contingent upon the instrument used and the type of feedback provided. The combination of haptic and auditory feedback consistently produced the most favorable outcomes, particularly in terms of timing and duration accuracy. Conversely, haptic feedback alone was insufficient for more complex instruments, such as the ukulele, resulting in the least successful learning outcomes. These findings indicate that integrating both haptic and auditory feedback is crucial for achieving optimal rhythm learning across various instruments.

Following these results, \cite{Fang2023} investigated whether other kinds of stimuli apart from vibration might be beneficial for \gls{phl} , exploring the effectiveness of different tactile sensations in \gls{phl} for piano songs. The study aimed to determine whether affective tactile sensations are as effective as the more commonly used discriminative sensation of vibration in teaching motor tasks through \gls{phl}. Additionally, it examined whether there are differences in learning rates and user perceptions across these tactile modalities. Each participant learned three different note sequences using the three tactile systems (vibration, tapping, and stroking) in a within-study design. After a 30-minute \gls{phl} session for each system, participants were tested on their ability to recall and play the sequences. The results indicated no significant differences in the effectiveness of the three tactile systems for \gls{phl}, confirming that all three modalities—vibration, tapping, and stroking—are effective for teaching piano sequences. However, tapping and stroking were found to be slightly more effective than vibration, with participants learning more notes and making fewer errors (up to $1.06$ fewer errors on average) when using these affective sensations. Additionally, over $50\%$ of participants rated stroking as the most pleasant sensation, while only $11\%$ favored vibration. The study also highlighted that \gls{phl} is particularly beneficial for inexperienced users, consistent with previous research. However, the overall accuracy rate per note sequence was lower than in earlier studies, possibly due to the absence of recall aids during the testing phase, such as auditory or visual cues.

\subsection{Typing Skills}
Building on the success of \gls{phl} in the \gls{mmt} project \cite{Markow2010, Kohlsdorf2010, Huang2010}, Seim et al. extended their research to explore its application in typing systems, aiming to use these findings as a foundation for passively teaching Braille typing \cite{Seim2014}. They investigated whether complex skills, such as typing and chord recognition, could be effectively taught through tactile interfaces using \gls{phl}, and also examined the role of visual feedback in the learning process.
Participants in the study were taught typing skills using fingerless gloves embedded with vibration motors. During practice sessions, audio cues were followed by corresponding vibration patterns to stimulate finger presses. Participants then typed the patterns into software that displayed either the actual letters they typed or asterisks as feedback. The study aimed to determine whether visualizing the letters during typing would aid or impede learning.
The results showed that visual feedback, where participants saw the letters they typed, actually hindered their performance compared to using asterisks or vibration only. Those using vibration-only feedback achieved better accuracy, though their typing speed was lower. This suggested that visual prompts may not be effective for \gls{phl}-based typing and that users might benefit more from audio prompts.
The study also found that the random presentation of letters and words did not significantly impact learning, highlighting the importance of session length and information chunk size. However, teaching chords through \gls{phl} was unsuccessful, as participants had difficulty distinguishing which fingers were being stimulated. The researchers concluded that accuracy, rather than speed, should be the primary metric for evaluating \gls{phl} effectiveness.
In a follow-up experiment, a \gls{phl} session without active practice was conducted, where participants played a memory card game while receiving training. The results indicated that participants could type with a low error rate and maintain consistent performance, successfully learning to use both hands with \gls{phl}. Some even achieved perfect accuracy on new phrases, though at reduced typing speed. This suggested that \gls{phl} can effectively teach typing skills, including the mapping of letters to keys.
Overall, the study concluded that \gls{phl} is a valuable method for teaching typing skills, particularly when using audio prompts rather than visual feedback. However, its effectiveness in teaching more complex tasks, such as chord recognition, remains limited.

To further investigate the potential of \gls{phl}, Seim expanded their research to determine whether it could be applied to a multi-row keyboard rather than the traditional one-finger-to-one-key mapping \cite{Seim2017}. The study focused on enhancing the speed and accuracy of a motor task, specifically numeric entry on a randomized keypad. The aim was to assess whether \gls{phl} could improve typing speed and reduce reliance on visual cues in a text entry system.
The researchers conducted a study using a 4x3 numeric keypad with a randomized key mapping, designed for use with the right hand only. The study included a pretest to establish a baseline, followed by multiple sessions in which participants alternated between a distraction task and a typing test. During the distraction task, participants in the \gls{phl} group received passive tactile stimuli corresponding to one row of the keypad at a time, which facilitated chunking and improved spatial memory.
The results demonstrated that participants in the \gls{phl} group significantly improved their typing speed compared to the control group. Additionally, \gls{phl} users looked at the keyboard significantly less during the task, indicating improved familiarity with the key layout and reduced dependency on visual cues. This suggested that \gls{phl} can effectively convert tactile stimuli into motor movements and enhance performance in text entry systems.
A pilot study included an additional test where participants’ hands were covered by a paper screen to assess their knowledge of the keypad layout without visual assistance. This further confirmed the effectiveness of \gls{phl} in reinforcing the spatial memory needed for typing.
The study concludes that \gls{phl} can be a powerful tool for improving motor task performance, particularly in learning and mastering keyboard typing skills. The use of wearable computing devices that provide passive tactile feedback presents a promising solution for training and enhancing the speed and accuracy of text entry tasks.

\subsection{Braille Learning}

Building on the initial success with typing, Seim et al. \cite{Seim2014a} explored the effectiveness of \gls{phl} in teaching Braille typing and reading through wearable technology. The study aimed to determine whether \gls{phl} could reduce errors in Braille typing and enhance the recognition and reading of Braille letters compared to traditional learning methods.
Participants in the study were passively taught the full Braille alphabet over several sessions using a wearable device that provided haptic feedback. The instruction method utilized a chorded input system based on sequential tapping patterns, a technique that had previously been ineffective with non-sequential patterns \cite{Seim2014a}. Participants engaged in tactile and visual Braille letter identification tasks, along with a distraction task to rigorously assess their learning.
The results demonstrated that participants who received passive haptic instruction exhibited a significant reduction in typing errors when typing phrases in Braille compared to those who did not receive haptic feedback. Moreover, participants were able to recognize and read more Braille letters from the phrases they typed, achieving a high recognition rate of the entire Braille alphabet by the end of the study. These findings suggested that \gls{phl}, facilitated by wearable technology, is a feasible and effective method for teaching Braille typing and reading.
The study also revealed that \gls{phl} allowed participants to learn words and complete their learning more quickly than those who did not receive haptic feedback. This indicates that typing practice in this context may also serve as effective reading practice, further enhancing the learning experience. The researchers concluded that \gls{phl} could be a valuable tool in Braille education, offering a passive yet powerful means of learning complex text entry skills.

Following this success, Caulfield et al. \cite{Learning2024} investigated the impact of a \gls{phl} glove on the learning rate, proficiency, and recall rate in Braille learners. The study compared the effectiveness of the glove-based \gls{phl} method with traditional memorization approaches.
The study was conducted in three phases. In the first phase, participants used flashcards to associate letters with Braille cell orientations, serving as a baseline for evaluating their initial proficiency without haptic feedback. The second phase involved a typing exercise using a keyboard, with some participants receiving haptic feedback through the glove while typing. The third phase consisted of a recall test conducted both with and without haptic feedback, followed by a retention test conducted several days later.
The results indicated no statistically significant difference in the effectiveness of using \gls{phl} via the glove compared to traditional memorization methods. While the glove provided haptic feedback, it did not significantly enhance learning outcomes. In some cases, the glove even appeared to hinder performance, as suggested by longer response times and lower recall rates in participants who used the glove. The study concludes that the \gls{phl} system tested may not be an effective tool for improving Braille learning.

\subsection{Morse Code}

In $2016$, Seim et al. investigated whether \gls{phl} could teach rhythm by using Google Glass to passively teach Morse code through head-based vibrations \cite{Seim2016a}. The study focused on whether Morse code, a rhythm-based system, could be learned passively through vibrations on the head rather than through hands, which are typically used in haptic feedback studies.
Participants were randomly assigned to either a control group or a \gls{phl} group. The \gls{phl} group received rhythmic haptic feedback via the Google Glass's \gls{bct} while learning Morse code, while the control group only heard the words without any Morse code information. Over four hours, participants engaged in various learning and distraction tasks, including typing and perception tests, to assess their Morse code proficiency.
The results showed that \gls{phl} could effectively teach Morse code using head-based vibrations. The \gls{phl} group achieved 94\% accuracy on a pangram typing task, with most participants reaching 100\% accuracy by the end of the study. They were also able to recognize and reproduce Morse code rhythms with minimal errors, demonstrating that the rhythm-based nature of Morse code could be learned through head-based haptic feedback. The \gls{phl} group outperformed the control group, showing a lower error rate and increased typing speed, which improved from $2.5$ words per minute (WPM) to $4$ WPM, approaching the target speed of $10$ WPM.
The study concluded that \gls{phl} is a viable method for teaching rhythm-based non-motor skills like Morse code through \gls{bct} devices like Google Glass. It highlighted the potential of \gls{phl} for eyes-free, silent text entry on mobile devices, offering new possibilities for learning and communication. However, it also noted that some active learning might occur when visual feedback is provided during testing, suggesting that future research should further isolate and measure the effects of \gls{phl} alone.

After the success Seim et al. explored in \cite{Seim2018} the potential of using smartwatch haptics to facilitate \gls{phl} of new skills, specifically focusing on Morse code. The study aimed to determine whether the subtle haptic feedback typically used for message alerts on smartwatches is sufficient for teaching skills and to compare the effectiveness of different durations of passive stimulation.
The researchers conducted a study where participants used a smartwatch to deliver low-amplitude haptic stimuli corresponding to Morse code. Participants underwent a pre-test to assess their initial knowledge of Morse code, followed by a period of \gls{phl} where they received haptic feedback while engaged in a distraction task. The study was designed as a between-subject experiment, with participants split into two groups: one receiving 8 minutes of passive stimulation per word and the other receiving 16 minutes.
The results demonstrated significant improvements in participants' ability to recall and recognize Morse code from pre-test to post-test, with those in the 16-minute stimulation group showing a 25-75\% improvement in accuracy compared to the 8-minute group. This suggested that extended exposure to haptic feedback enhances learning outcomes. A follow-up recall test administered 1-3 days later indicated that the participants retained the information learned during the study.
The study concludes that smartwatches, despite their low-amplitude actuators, can effectively support \gls{phl} and help users learn new skills like Morse code. The findings suggested that longer durations of haptic stimulation may lead to better retention and performance, indicating that smartwatch haptics could be a viable tool for skill acquisition through \gls{phl}. This research opened up new possibilities for using everyday wearable devices for educational and training purposes.

However, \cite{Pescara2019} suggested that the training and testing procedures used in earlier studies may have leaked information to participants, leading to inadvertent active learning. They revisited and critiqued the design choices of previous studies such as \cite{Seim2016a, Seim2018}, highlighting potential flaws that might have facilitated active learning \cite{Pescara2019}.
To address this, \cite{Pescara2019} investigated whether \gls{phl} could effectively teach Morse code without requiring active attention from learners, building on prior research. The study divided participants into five groups to explore the effectiveness of \gls{phl} in teaching Morse code. Participants were exposed to vibration patterns on a wristband, each corresponding to a Morse code character, while also engaging in distraction tasks of varying difficulty to simulate divided attention.
The results showed that while it is possible to learn Morse code passively through haptic feedback, learning rates were significantly lower compared to when active attention was involved. The findings suggested that \gls{phl} can facilitate the acquisition of simple motor skills with minimal attention but is less effective for complex, non-motor tasks like learning Morse code, which requires declarative memory engagement. The study also found that participants performed better when feedback was provided during learning, though the difficulty level of the distraction tasks did not significantly affect outcomes.

\subsection{Multi-limb Rhythm Learning}
In $2011$, the potential of \gls{phl} for multi-limb skills was further explored. Bouwer et al. \cite{Bouwer2011} introduced the Haptic Ipot \cite{Holland2010, Bouwer2011}, a system designed to teach drumming techniques. In this study, participants wore elastic Velcro bands equipped with haptic vibrotactile devices that delivered rhythmic stimuli to different limbs. The stimuli were played back silently, and during the haptic feedback sessions, participants engaged in a reading comprehension task, similar to those used in previous studies \cite{Huang2008, Huang2010}.
Participants were exposed to two different rhythms, and their baseline performance was measured using a MIDI drum kit. Following the \gls{phl} session, participants were tested again on the MIDI drum kit to evaluate their accuracy, timing, number of attempts, and errors during their best attempt. A questionnaire was also administered to collect subjective feedback on the learning experience.
The preliminary results suggested that \gls{phl} of multi-limb rhythms is a promising approach. Participants demonstrated improvements in both rhythm accuracy and timing, indicating that even while focusing on other tasks, individuals can still absorb and reproduce complex rhythmic patterns through haptic feedback.

\subsection{Skin Reading}

\Gls{vsr} \cite{Luzhnica2016} is a method of encoding vibrotactile patterns to represent symbols, which can be combined to convey complex messages such as words and phrases \cite{Luzhnica2018}. Building on their previous research on VSR, Luzhnica et al. \cite{Luzhnica2018} explored the effectiveness of \gls{phl} as a training method for skin reading, where information is conveyed through vibrotactile patterns. This study specifically investigated whether \gls{phl} can be used to teach participants to comprehend text transmitted via these patterns and examines whether the speed of transmission affects recognition accuracy.
The researchers conducted an experiment in which participants underwent 30 minutes of training to learn 10 letters of the German alphabet encoded into vibrotactile patterns. The training utilized a glove equipped with six vibromotors placed on the back of the hand, replicating the design from \cite{Luzhnica2016}. Each letter was encoded by activating one or two vibromotors in an \gls{ost} sequence, which is known to provide better perception than simultaneous activation of multiple motors.
During the training, participants engaged in a distraction task, such as playing a game\footnote{The snake game \url{https://en.wikipedia.org/wiki/Snake_(video_game_genre)}}, while passively receiving vibrotactile cues. Each letter's pattern was repeated multiple times over 12 rounds of training. Additionally to the pre- and post-tests, the study conducted a recall test the following day to assess the participants' ability to reconstruct and recognize the letters they had learned \cite{Luzhnica2018}.
The results demonstrated that participants could recall the learned vibrotactile patterns and accurately reconstruct and recognize the letters both immediately after training and one day later. The accuracy of recognition and reconstruction was consistent with previous research, indicating that \gls{phl} is an effective method for training skin reading. Additionally, the study found no significant difference in comprehension accuracy across different transmission speeds, suggesting that participants could accurately comprehend the information transmitted at varying speeds.
And, henceforth, concludes that \gls{phl} is a promising and effective method for training vibrotactile skin reading, offering a viable alternative to more demanding and time-consuming active training methods. The ability to comprehend information regardless of transmission speed further enhances the potential applications of \gls{phl} in this domain.


\subsection{Effects of PHL Practice in Rehabilitation}
In their work on the \gls{mmt} framework in $2010$, Markow et al. explored potential improvements in dexterity and sensation for patients with incomplete \gls{sci} while learning a new skill set. This research led to their second \gls{mmt} paper \cite{Markow2010}, which presented a pilot study on hand rehabilitation in individuals with tetraplegia due to \gls{sci}.
The study demonstrated that \gls{mmt} could stimulate afferent nerves, potentially improving motor function and sensory perception through the active learning (piano practice) mode. Participants with tetraplegia indicated progress in sensory tests such as the Semmes-Weinstein monofilament evaluation and the Grasp and Release Test (GRT) after four weeks of practice, consisting of three 30-minute sessions per week. Additionally, three different glove designs were tested: a golf-style glove, an open-flap glove, and a Velcro-finger glove (as depicted in \autoref{fig:glove_designs}).
The preliminary findings suggested that \gls{mmt} holds promise as a form of hand rehabilitation for individuals with tetraplegia resulting from incomplete \gls{sci}. The work presented in this paper lays the foundation for future studies to evaluate the broader applicability of \gls{mmt} in the tetraplegia population.


\subsection{Stenography Learning}

Aveni et al. \cite{Aveni2019} explored passive haptic learning (\gls{phl}) for stenography training, focusing on a system that combines tactile feedback with spatial tasks. The study aimed to teach participants to use a stenotype keyboard, where each key corresponds to an English sound. Over four 10-minute sessions, participants wore two-handed fingerless gloves with a vibration motor on the dorsal side of each hand and played the game "SpikeDislike2" as a distraction\footnote{\url{https://gamejolt.com/games/spikedislike2/}}.
The training focused on combining “subchords,” or word-parts, to form unfamiliar words. Each finger was responsible for two keys, except the left pinky, and the system used temporal offsets in the vibration patterns to stimulate the fingers from left to right and top to bottom. A distinct tapping rhythm indicated whether the top, bottom, or both keys needed to be pressed simultaneously.
Results showed a significant improvement in typing accuracy compared to the control group, based on uncorrected error rates.
The study demonstrated that passive tactile feedback can help novices learn stenography, even without explicit instruction on combining subchords. Participants autonomously figured out word-formation through the modular training structure, with errors mainly being horizontal or vertical, indicating correct finger placement. This suggests \gls{phl} as an effective and intuitive method for teaching complex skills like stenography.
